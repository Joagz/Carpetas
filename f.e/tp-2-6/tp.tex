\documentclass[a4paper,11pt]{article}
\usepackage[utf8]{inputenc}
\usepackage[spanish]{babel}

\title{Actividad: Derechos Humanos}
\date{2 de junio, 2025}
\author{Joaquín Gómez}

\begin{document}
\maketitle

\begin{enumerate}
    \item \textbf{¿Se cumplen los DDHH en la práctica? Fundamenta.} \\
    En la practia, muchos paises no \textbf{garantizan} los derechos humanos. La desocupación y la pobreza, el trabajo precario, la explotación laboral, los derechos de los niños, entre otros, no son respetados en la mayoría de países. Que no se garanticen los DDHH no significa que se violen, o que no sean castigados penalmente.

    \item \textbf{Explica la relación que existe entre pobreza, desigualdad y políticas neoliberales.} \\
    Las políticas neoliberales impulsan el libre mercado y la reducción de asistencia social desde el Estado. La relación con la pobreza se da en que no se proporciona ayuda desde el Estado a las personas en situación de vulnerabilidad. Lo cual, puede considerarse una violación a los derechos humanos.

    \item \textbf{¿En qué consisten los derechos de los niños?} \\
    \textbf{Derecho a la vida:}
    Todo niño tiene derecho a la vida y debe ser protegida. 
    
    \textbf{Derecho a la identidad:}
    Los niños tienen derecho a un nombre, nacionalidad y a conocer a sus padres. 
    
    \textbf{Derecho a la familia:}
    Los niños tienen derecho a crecer con sus padres y a tener una familia alternativa cuando sea necesario. 
    
    \textbf{Derecho a la educación:}
    Los niños tienen derecho a una educación pública y gratuita en todos los niveles, según la ley argentina. 
    
    \textbf{Derecho a la salud:}
    Los niños tienen derecho a atención médica y a un entorno seguro y saludable. 
    
    \textbf{Derecho a la protección:}
    Los niños tienen derecho a protección contra todo tipo de abuso, violencia y negligencia. 
    
    \textbf{Derecho a la expresión:}
    Los niños tienen derecho a expresar sus opiniones y a ser escuchados, especialmente en los temas que les afectan. 
    
    \textbf{Derecho al juego:}
    Los niños tienen derecho a jugar y a disfrutar de actividades recreativas que promuevan su desarrollo. 
    
    \textbf{Derecho a la recreación:}
    Los niños tienen derecho a un tiempo libre adecuado y a la recreación. 
    
    \textbf{Derecho a la intimidad:}
    Se debe respetar la vida privada de los niños, incluyendo su hogar y comunicaciones

    \item \textbf{¿Qué políticas llevó a cabo el Estado argentino para tratar de erradicar la violencia contra la mujer? ¿Qué problemáticas sufren actualmente las mujeres?} \\
    El Estado argentino ha implementado leyes como la Ley 26.485 de protección integral contra la violencia hacia las mujeres, el Programa Acompañar, y campañas de concientización. Sin embargo, las mujeres siguen enfrentando femicidios, violencia doméstica, desigualdad laboral y acoso. La respuesta estatal muchas veces es insuficiente y lenta.
\end{enumerate}

\end{document}
