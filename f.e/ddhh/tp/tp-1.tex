\documentclass[a4paper, 12pt]{article}
\usepackage[spanish]{babel}
\usepackage[a4paper,margin=1in]{geometry}

\title{Derechos Humanos en Argentina}
\author{Tironi Constanzo --- Gómez Joaquín}
\date{$5^{to}$ ``A''}

\begin{document}
\maketitle
\pagebreak

\section{Derechos de los niños}

Los Derechos Humanos se fundamentan en la idea de que todos los humanos somos
iguales por la condición de ser humanos. En este caso, cuando un individuo es
tratado de forma desigual, se produce una \textbf{discriminación} (ya sea
positiva o negativa).

Sin embargo, se reconoce la existencia de grupos con características
desfavorables. Se dice entonces que estos grupos se encuentran en situación de
\textbf{vulnerabilidad}.

En estos casos se realiza una distinción, o \textbf{discriminación positiva},
en pos de contrarrestar las desventajas que poseen estos grupos.

En 1924 se realiza la Declaración de Ginebra sobre los Derechos del Niño. Ésta
fue el primer instrumento internacionale que señaló las necesidades especiales
de los niños y los deberes que los adultos debían realizar para asistirlos.

La ONU en 1959, continuó con el tratado del problema de la vulnerabilidad de
los niños. Por ello, proclamó la \textbf{ Declaración de los Derechos del
    Niño}, que consideró que los niños y niñas, a causa de su madurez física y
mental, necesitan de protección y cuidados especiales.

Al tratarse de declaraciones, solo se definen ideas, pero no se establecen
compromisos de tratamiento obligatorio. En 1989, la Asamblea de las Naciones
Unidas aprobó la \textbf{Convención Internacional sobre los Derechos del Niño
    (CIDN)}. La CIDN define como ``niño'' a todo individuo menor de 18 años.

La aprobación de la CIDN significó un cambio sustancial en la forma de concebir
la infancia. Antes, se pensaba a los menores como ``incapaces'', y que eran
objetos de protección, receptores de asistencia. A partir de la CIDN, los niños
pasan a ser sujetos de derecho.

\section{Perspectiva general}

Al terminar la Segunda Guerra Mundial, se produjo una expansión de los Derechos
Humanos. Se declaró que los DDHH son irrenunciables, y que, por ende, violarlos
implica un delito de lesa humanidad.

Se crean distintas instituciones encargadas de investigar y juzgar las
violaciones de DDHH.

\section{Argentina y los Derechos Humanos}

Argentina sufrió una violación sistemática de los Derechos Humanos durante la
década de 1970. Tras terminar la dictadura, se hizo visible el valor del
derecho a la vida por encima de cualquier disputa por diferencia política,
económica, social, etc.

El contexto social y económico de Argentina, provoca que no todos los DDHH
puedan ser respetados en su totalidad. La precarización laboral, de la
vivienda, el hambre, la explotación laboral, y la discriminación étnica, siguen
siendo problemas que afectan a la sociedad.

\subsection*{Terrorismo de Estado}
La doctrina de Seguridad Nacional se dio durante la década de 1970, y fue
establecida por EEUU, sostenía que el comunismo estaba por todas partes y que
los estados latinoamericanos debían actuar e conjunto con sus fuerzas armadas
para erradicar el comunismo que se gestaba dentro de las fronteras, formando un
``enemigo interno''.

\end{document}
