\documentclass{article}
\title{El papel del Estado y Organismos internacionales frente a los Derechos Humanos}



\begin{document}

\maketitle

\section{Actividad}

\begin{enumerate}
      \item ¿Cuál es la responsabilidad del Estado en materia de DDHH?
            ¿Cuándo se habla de violación de los DDHH?
      \item ¿Cuál es la finalidad de los sistemas de protección de DDHH?
            ¿Cuáles son sus elementos?
      \item ¿Qué es la OEA?¿Cuáles son sus instituciones, que intervienen en cuestiones de DDHH?
            ¿Cuál es la función de la corte interamericana de DDHH?
      \item ¿Qué implica un tratado sobre DDHH?
\end{enumerate}

\begin{enumerate}
      \item Los Estados son responsables de garantizar los Derechos Humanos a los
            habitantes que se encuentren dentro de su territorio. Entre sus obligaciones se
            encuentran:
            \begin{itemize}
                  \item Respetar los DDHH
                  \item Gerantizar el ejercicio pleno de los DDHH
                  \item Promover su concreción efectiva
            \end{itemize}

            Se habla de violación de los DDHH cuando es un Estado, organismo público o
            institución perteneciente al mismo quien vulnera un derecho. En el caso de que
            una institución u organismo privado vulnere un derecho humano, se trata de un
            delito.
      \item Los sistemas de protección de los Derechos Humanos tienen como objetivo
            asegurar que los Estados respeten los mismos y, al mismo tiempo, otorgar una
            serie de mecanismos para obtener justicia y reparación en aquellos casos en que
            se hayan vulnerado.

            Esta serie de herramientas está formada por:
            \begin{itemize}
                  \item Las \textbf{normas}, que reconocen los derechos de las personas y los imponen a
                        los Estados.
                  \item Los \textbf{órganos}, que se encargan de aplicar esas normas.
                  \item Los \textbf{procedimientos}, que es el modo y los pasos a seguir para la
                        aplicación de las normas, que garantizan la transparencia e imparcialidad de
                        las mismas.
            \end{itemize}

      \item La OEA es la Organización de los Estados Americanos, rige el Sistema
            interamericano de Derechos Humanos, actualmente la integran los 35 Estados
            independientes del continente americano. Su propósito es lograr en sus Estados
            miembros, un orden de paz y justicia, fomentar la solidaridad, robustecer su
            colaboración y defender su soberanía.

            Por otro lado, los órganos de la OEA que intervienen en cuestiones de DDHH son
            la \textbf{Comisión Interamericana de Derechos Humanos} y la \textbf{Corte
                  Interamericana de Derechos Humanos}

            La Corte Interamericana de Derechos Humanos es una institución judicial que
            vela por la aplicación e interpretación de la Convención Americana. Cuando un
            hecho de violación de los DDHH ocurre y es reconocido por esa Convención, y al
            presentarse en la justicia de ese país no se resuelve satisfactoriamente, se
            puede acudir a la Comisión Interamericana de DDHH, si es admisible, se
            transfiere a la Corte.

      \item Un tratado sobre derechos humanos implica un acuerdo legal entre Estados para
            reconocer, proteger y garantizar ciertos derechos fundamentales de las
            personas. Al firmarlo y ratificarlo, un Estado: 
      
            \begin{itemize}         
                  \item Se compromete legalmente a cumplir con las obligaciones que el tratado establece.
                  
                  \item Debe adaptar sus leyes y políticas internas para garantizar esos derechos.
                  
                  \item Puede ser supervisado por organismos internacionales que vigilan el cumplimiento del tratado.
                  
                  \item Permite que personas o grupos denuncien violaciones de derechos ante instancias internacionales (si el tratado lo permite).
            \end{itemize}
      \end{enumerate}

\end{document}

