\documentclass{report}
\usepackage[spanish]{babel}
\usepackage{tikz}
\usetikzlibrary{positioning}

\title{Industrias}
\author{Joaquín Gómez}

\begin{document}
\maketitle

\chapter{Industria}
\textit{Es una actividad cuyo propósito es transformar las materias primas en productos
    elaborados, semi-elaborados o super-elaborados utilizando una fuente de
    energía}. Para su desarrollo necesitan materiales, maquinarias y recursos
humanos, organizados habitualmente en empresas por su especialización laboral.
La industria es la actividad económica fundamental del sector
\textbf{secundario}

\section{Tipos de organizaciones industriales}

\begin{itemize}
    \item \textbf{Industria pesada}: transforma materias primas en productos elaborados de gran tamaño, por ejemplo, maquinarias.
    \item \textbf{Siderúrgicas}: industria que produce aceros y derivados del hierro
    \item \textbf{Metalúrgicas}: trabaja metales diferentes del hierro, como por ejemplo el aluminio
    \item \textbf{Cementeras}: fabrican cemento para la construcción a partir de piedra caliza y otros materiales
    \item \textbf{Químicas}: elaboran productos mediante procesos químicos
    \item \textbf{Petroquímicas}: transforman derivados del petrato en productos químicos
    \item \textbf{Automovilísticas}: producen vehículos y autopartes
    \item \textbf{Industria ligera}: fabrican bienes de consumo final para el público
    \item \textbf{Alimenticia}: transforman materias primas en productos comestibles
    \item \textbf{Industria de punta}: usa tecnología avanzada y están en constante innovación
    \item \textbf{Farmacéutica}: producen medicamentos y fármacos
    \item \textbf{Textil}: fabrican tela, ropa y fibras
    \item \textbf{Armamentística}: desarrollan armamento militar y equipos
    \item \textbf{Robótica}: diseña y construye robots y sistemas automatizados
    \item \textbf{Informática}: produce hardware y software
    \item \textbf{Papelera}: fabrica papel a partir de madera
    \item \textbf{Astronáutica}: desarolla vehículos y tecnología para el espacio
    \item \textbf{Mecánica}: crea maquinaria y equipos industriales
    \item \textbf{Aeroespacial}: fabrica aviones, satélites y tecnología aérea y del espacio
\end{itemize}

\chapter{Organizaciones Industriales}

La principal característica de las empresas industriales es dedicarse a la
transformación de productos, obliga a que su organización interna difiera
considerablemente de la adoptada por las empresas comerciales.

\section{Ciclo operativo}

Es el conjunto de actividades básicas que realiza cualquier empresa para
alcanzar sus objetivos

\subsection{Ciclo operativo de una empresa comercial}

\begin{center}
    \begin{tikzpicture}
        \matrix [column sep=7mm, row sep=5mm] {
            \node (me) [draw, shape=rectangle] {Compra (mercaderías)};                                     \\
            \node (pa) [draw, shape=rectangle] {Pagos (de las mercaderías)};                               \\
            \node (ve) [draw, shape=rectangle] {Ventas (de las mercaderias en el mismo estado adquirido)}; \\
            \node (co) [draw, shape=rectangle] {Cobro (a los clientes o deudores)};                        \\
        };
        \draw[<-, thick] (pa) -- (me);
        \draw[<-, thick] (ve) -- (pa);
        \draw[<-, thick] (co) -- (ve);
    \end{tikzpicture}
\end{center}

\begin{itemize}
    \item Vende los bienes como los adquiere
    \item Los bienes adquiridos se denominan \textbf{mercaderías}
    \item Los bienes vendidos también se denominan \textbf{mercaderías}
    \item Cuenta con una contabilidad general que brinda información interna y externa
\end{itemize}

\begin{center}
    \begin{tikzpicture}
        \matrix [column sep=7mm, row sep=5mm] {
            \node (me) [draw, shape=rectangle] {Compra (materias primas, insumos y materiales)}; \\
            \node (pa) [draw, shape=rectangle] {Pagos (de las mismas a los proveedores)};        \\
            \node (fa) [draw, shape=rectangle] {Fabricación o elaboracion de productos};         \\
            \node (ve) [draw, shape=rectangle] {Ventas (de los productos elaborados)};           \\
            \node (co) [draw, shape=rectangle] {Cobros (a clientes, deudores)};                  \\
        };
        \draw[<-, thick] (pa) -- (me);
        \draw[<-, thick] (fa) -- (pa);
        \draw[<-, thick] (ve) -- (fa);
        \draw[<-, thick] (co) -- (ve);
    \end{tikzpicture}
\end{center}

\textbf{Características}
\begin{itemize}
    \item Vende los bienes después de su transformación
    \item Los bienes adquiridos son ``materia prima''
    \item Los bienes que se venden se denominan ``productos elaborados''
    \item Cuentan con una contabilidad general que brinda información interna y externa.
          Y con una contabilidad de costo que brinda información de uso interno.
\end{itemize}

\begin{center}
    \begin{tikzpicture}
        \node (in) [draw, shape=rectangle] {INDUSTRIALIZACIÓN};

        \matrix [column sep=7mm, row sep=25mm] {
            \node (im) [draw, shape=rectangle] {Imperios Industriales};     &
            \node (ex) [draw, shape=rectangle] {Expansión Urbana};            \\
            \node (am) [draw, shape=rectangle] {Amenazas Medioambientales}; &
            \node (bu) [draw, shape=rectangle] {Burgueses y Proletariados};   \\
        };
        \draw[<-, thick] (im) -- (in);
        \draw[<-, thick] (ex) -- (in);
        \draw[<-, thick] (am) -- (in);
        \draw[<-, thick] (bu) -- (in);
    \end{tikzpicture}
\end{center}

La industrialización se refiere a la producción de bienes en grandes
proporciones y también alude al proceso mediante el cual una sociedad o Estado
pasa de una economía agrícola a una economía industrializada.

La industrialización se genera en un sector específico y se fundamenta en el
desarrollo de maquinarías, técnicas y procesos de trabajo con el fin de
producir más en menos tiempo, así como en el crecimiento económico que busca
maximizar los beneficios y los resultados del PBI (Producto Bruto Interno).

Gracias a la industrialización, se dio origen a un nuevo orden social,
económico, político, cultural y geográfico.

Los trabajos agrícolas se sistematizaron con el desarrollo de nuevas
maquinarias, los habitantes del campo emigraron a las nuevas y grandes ciudades
en busca de oportunidades de empleo, mejores salarios, un nuevo hogar, mayor
calidad de vida, se estandarizó la familia nuclear y no numerosa, entre otras.

\end{document}