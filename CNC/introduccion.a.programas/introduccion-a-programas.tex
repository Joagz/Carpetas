\documentclass{article}
\usepackage{amsmath}
\usepackage{tikz}
\usepackage[spanish]{babel}
\title{Introducción a programas --- Taller de CNC}
\author{Joaquín Gómez --- $5^{to}$ ``A''}
\date{\today}

\begin{document}
\maketitle
\section{Llamar a un programa}
\begin{itemize}
    \item Cambiar al modo EDICIÓN
    \item Pulsar la tecla $PRGRM$
    \item Con la tecla de software $BIBLIO$ se ven los programas existentes
    \item Escribir el ID del programa (por ej. $O777$)
    \item Para un nuevo programa, pulsar $INSRT$
    \item Para un programa ya existente pulsar { $\downarrow$ }
\end{itemize}

\section{Estructura de un programa}

\begin{itemize}
    \item $NO1,\,NO2,\,NO3,\,NO4,\,NO5,\,NO6,\,NO7 \longrightarrow$ Instrucciones Técnicas
    \item $T0101,\,M06,\,S1200,\,M03,\,F1 \longrightarrow$ Instrucciones Tecnológicas
    \item $G99,\,G54,\,G00,\,X20,\,Z1,\,G01,\,Z22 \longrightarrow$ Instrucciones Geométricas
\end{itemize}

\begin{equation*}
    \begin{matrix}
        NO1 & G99   &     &    \\
        NO2 & G54   &     &    \\
        NO3 & T0101 & M06      \\
        NO4 & G97   &     &    \\
        NO5 & S1200 & M03 &    \\
        NO6 & G00   & X20 & Z1 \\
        NO7 & G54   & Z22 & F1 \\
    \end{matrix}
\end{equation*}

\section{Datos del maquinado}
\begin{center}
    $\textnormal{Herramienta} : \textnormal{Tool} = \textbf{T}$
\end{center}
\begin{equation*}
    \begin{matrix}
        T0101 & \textnormal{Llama a la herramienta}                     \\
        T0202 & \textnormal{Número $\dots$ con su corrector respectivo} \\
        T0303\end{matrix}
\end{equation*}

\section{Setup o seteo de la herramienta}

\textbf{Setup geométrico en el eje $x$:}
\begin{itemize}
    \item Elegir el MDI de la herramienta, además seleccionar el sentido y velocidad de
          giro
    \item Realizar un cilindrado manualmente y retrirar sin modificar las coordenadas $x$
    \item Parar el plato con ``SPDL STOP''
    \item Medir con un calibre el diámetro
    \item OFFSET --- COMP (Compensación) --- GEOME (Geométrica)
    \item Ubicarse en el $x$ de la herramienta
    \item Cargar el valor medido. Por ejemplo, diámetro 30 $\longrightarrow$ cargar $x30$
    \item Luego ``+Media''
    \item En MDI cargar la herramienta con la compensación
\end{itemize}

\textbf{Setup geométrico en el eje $z$:}
\begin{itemize}
    \item Elegir el MDI de la herramienta, además seleccionar el sentido y velocidad de
          giro
    \item Realizar un cilindrado manualmente hasta una profundidad de $1mm$
          aproximadamente $x$
    \item Parar el plato con ``SPDL STOP''
    \item Medir con un calibre el diámetro
    \item OFFSET --- COMP (Compensación) --- GEOME (Geométrica)
    \item Ubicarse en el $z$ de la herramienta
    \item Cargar el valor 0
    \item Luego ``Media''
    \item En MDI cargar la herramienta con la compensación
\end{itemize}

\section{Los grupos «G» y «M»}

El torno funciona mediante la ejecución de órdenes de desplazamiento y de
condiciones de entorno. Las órdenes de desplazamiento correspnden a las
funciones G, denominadas \textbf{funciones preparatorias}, que tienen relación
directa con los movimientos de la herramienta, así como el desbastado de la
pieza de trabajo. Por su parte, las funciones M entregan las
\textbf{condiciones auxiliares} en que se trabajará (con o sin lubricante,
sentido de giro del husillo, etc.). Para la ejecución de un programa cualquiera
deben activarse varias funciiones G y M, las cuales se dividen en grupos, según
el tipo de acción que representen.

\section{Descripción breve de las direcciones}

\begin{itemize}
    \item Direcciones de desplazamiento $x$ y $z$: se refieren en forma absoluta a un
          origen de coordenadas. $z$ se mide paralelamente al eje de giro del husillo
          ($z$ negativo hacia la base del husillo), mientras que $x$ es la medida del
          diámetro ($x$ positivo por encima del eje de giro del husillo).

    \item Direcciones de desplazamiento $u$ y $w$: ídem $x$ y $z$ pero los
          desplazamientos se miden \textbf{incrementalmente} desde el punto de partida
          del movimiento. Son desplazamientos relativos. En este caso, $u$ \textbf{no
              representa} medidas diametrales, sino que es la distancia entre el punto
          inicial y el final (distancia radial).
\end{itemize}

\begin{center}
    \begin{tikzpicture}
        \draw[<-, thick] (0,2)--(0,0);
        \draw[<-, thick] (0,-2)--(0,0);

        \draw[<-, thick] (2,0)--(0,0);
        \draw[<-, thick] (-2,0)--(0,0);

        \draw (2.3,0) node{$Z^{+}$};
        \draw (-2.3,0) node{$Z^{-}$};
        \draw (0,2.3) node{$X^{-}$};
        \draw (0,-2.3) node{$X^{+}$};
    \end{tikzpicture}
    \begin{tikzpicture}
        \draw[<-, thick] (0,2)--(0,0);
        \draw[<-, thick] (0,-2)--(0,0);

        \draw[<-, thick] (2,0)--(0,0);
        \draw[<-, thick] (-2,0)--(0,0);

        \draw (2.3,0) node{$W^{+}$};
        \draw (-2.3,0) node{$W^{-}$};
        \draw (0,2.3) node{$U^{-}$};
        \draw (0,-2.3) node{$U^{+}$};
    \end{tikzpicture}
\end{center}

\section{Otras direcciones}

\begin{itemize}
    \item Dirección $F$: indica avance
    \item Dirección $S$: indica velocidad
    \item Dirección $R$: indica radio o empalme
    \item Dirección $A$: indica ángulo
    \item Dirección $C$: indica chaflán
\end{itemize}

\section{funciones G00 o G0}
\textbf{Los carros se desplazan a la velocidad máxima al punto final programado (posición de cambio de herramienta, punto inicial para el siguiente arranque de viruta).}
\begin{itemize}
    \item Mientras se ejecuta G00 se suprime un avance programado F
    \item La velocidad de avance rápido la define el fabricante de la máquina
    \item El interruptor de correción de avance está limitado al $100\%$
    \item Debe verificarse previamente que no haya obstáculos en el camino de la
          herramienta
    \item En el caso de modificar entre bloques ambas coordeanadas, la herramienta moverá
          los dos ejes a la vez
\end{itemize}

\section{Funciones G01 o G1}

Movimiento recto (refrenteado, torneado longitudinal, torneado cónico) con
velocidad programada de avance.

\begin{itemize}
    \item No realizará el movimiento si no está acompañado de un avance o si este no fue
          cargado previamente.
    \item Siempre que entre en contacto con el material debo emplear movimiento de
          trabajo.
    \item En caso de un cilindrado se modifica el eje $z$. Mientras que en el frenteado
          el eje $x$.
\end{itemize}

\section{Códigos G básicos.}
\subsection*{Descripción de comandos de funciones G}

\textbf{$G01$ interpolación lineal}. Formato: 
\begin{equation*}
\begin{matrix}
    N\dots & G01 & X(U)\dots & Z(W)\dots & F\dots
\end{matrix}
\end{equation*}

Movimiento recto (refrentado, torneado longitudinal, torneado cónico) con velocidad programada de avance.

Ejemplo: $G90$ Absoluto.
\begin{equation*}
    \begin{matrix}
        N\dots & G95 \\
        \dots & \dots \\
        N20 &G01 &X40 &Z20.1 &F0.1
    \end{matrix}
\end{equation*}

$G91$ Incremental.
\begin{equation*}
    \begin{matrix}
        N\dots & G95 \\
        \dots & \dots \\
        N20 &G01 &X 0 &Z-25.9
    \end{matrix}
\end{equation*}

\textbf{$G02$ interpolación circular a la derecha}
\textbf{$G03$ interpolación circular a la derecha}

Formato
\begin{equation*}
\begin{matrix}
    N\dots & G02 & X(U)\dots & Z(W)\dots& I\dots & K\dots
\end{matrix}
\end{equation*}
Alternativamente
\begin{equation*}
\begin{matrix}
    N\dots & G02 & X(U)\dots & Z(W)\dots& R\dots & F\dots
\end{matrix}
\end{equation*}

$X,Z,(U),(W) \longrightarrow$ Punto final del arco 
$I,K\longrightarrow$ Parámetros incrementales del arco. (Distancia desde el punto inicial al centro del arco (I), está en relación con el eje X, K con el eje Z).
$R \longrightarrow$ Radio del arco 

La herramienta se desplazará al punto final a lo largo del arco definido con el avance programado en F.



\section{Programas}

\subsection{Figura 1.}
\begin{equation*}
    \begin{matrix}
           & O0001 \\ 
        N10 & T0303 \\
        N20 & G97 & S1500 & M03 & F0.18 \\
        N30 & G0 & X0 & Z2 \\ 
        N40 & G1 & & Z0 \\
        N50 & & X10 &   Z{-8} \\ 
        N60 & & X10 &   Z{-17} \\
        N70 & & X20 &   Z{-17} \\
        N80 & & X20 &   Z{-28} \\
        N90 & & X30 &   Z{-36} \\
        N100 & & X30 &  Z{-45} \\
        N110 & G0 & X100 & Z100 \\
        N120 & M30 \\
    \end{matrix}
\end{equation*}


\subsection{Figura 2.}
\begin{equation*}
    \begin{matrix}
           & O0002 \\ 
        N10 & T0303 \\
        N20 & G97 & S1500 & M03 & F0.18 \\
        N30 & G0 & X0 & Z2 \\ 
        N40 & G1 & & Z0 \\
        N50 & &  X4 & Z0 \\ 
        N60 & &  X4 & Z{-3} \\
        N70 & &  X14 & Z{-9} \\
        N80 & &  X22 & Z{-9} \\
        N90 & &  X22 & Z{-18} \\
        N100 & & X30 & Z{-18} \\
        N110 & & X32 & Z{-19} \\
        N120 & & X32 & Z{-28.4} \\
        N130 & G0 & X100 & Z100 \\
        N140 & M30 \\
    \end{matrix}
\end{equation*}

\subsection{Figura 3.}
\begin{equation*}
    \begin{matrix}
        & O0003 \\ 
        N10 & T0303 \\
        N20 & G97 & S1500 & M03 & F0.18 \\
        N30 & G0 & X0 & Z2 \\ 
        N40 & G1 & & Z0 \\
        N50 & G03 & X16 & Z{-8} & R8 \\ 
        N60 & G1 & X20 & Z{-8} \\
        N70 & & X23 & Z{-9.5} \\
        N80 & & & Z{-20} \\
        N90 & G02 & X29 & Z{-23} & R3 \\
        N100 & G1 & X30 & Z{-23} \\
        N110 & G03 & X32 & Z{-25} & R2\\
        N120 & G1 & X32 & Z{-35} \\
        N130 & G0 & X100 & Z100 \\
        N140 & M30 \\
    \end{matrix}
\end{equation*}

\end{document}