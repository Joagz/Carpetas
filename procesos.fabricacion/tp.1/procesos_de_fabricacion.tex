\documentclass{article}
\usepackage[spanish]{babel}

\title{Trabajo Práctico: Procesos de Fabricación}
\author{Joaquín Gómez, $5^{to}$ `A'}

\begin{document}
\maketitle
\begin{enumerate}
    \item ¿De qué depende el éxito de muchas empresas en la actualidad?
    \item ¿Qué es un proceso de fabricación?
    \item ¿Para qué se utiliza el informe técnico?
    \item ¿Cuál es la estructura del informe técnico y por qué es necesario que tenga una?
    \item ¿Por qué crees necesario seguir pasos para la elaboración del informe técnico?
    \item ¿Qué es y para qué sirve la hoja de operaciones y la hoja de procesos?
\end{enumerate}

\pagebreak

\section{¿De qué depende el éxito de muchas empresas en la actualidad?}
En la actualidad, el éxito de las empresas recae en la capacidad que tienen
para generar productos, bienes o servicios de calidad y a un precio
competitivo. Su capacidad de adaptación a las nuevas tecnologías es fundamental
para que las empresas no caigan en la obsolescencia y puedan expandirse a
nuevos mercados.

\section{¿Qué es un proceso de fabricación?}
Se trata de un conjunto de procesos o actividades que involucran la
planificación, el diseño, la producción y administración, que tienen como
finalidad la creación de un producto.

\section{¿Para qué se utiliza el informe técnico?}
Un informe técnico es un documento que expone información específica acerca de
una situación, en este caso de un proceso, actividad, estado de un producto,
problema, etc. El mismo está dirigido ya sea a una persona, equipo, u empresa;
la finalidad del informe técnico es transmitir información detallada de un
proceso, para luego tomar decisiones: cómo mejorarlo, cómo lograr que el
proceso sea más eficiente, cómo solucionar el problema, etc.

\section{¿Cuál es la estructura del informe técnico y por qué es necesario que tenga una?}
El informe técnico tiene que estar estructurado de forma que el lector, una
persona calificada para entenderlo, pueda comprenderlo fácilmente, y redactar
un informe para solucionar la situación.

El informe debe contener una introducción, desarrollo o cuerpo, una conclusión
y los anexos.

\subsection{Introducción}
En ésta sección se resume el tema y se aclaran los objetivos. Se establece un
marco de referencia conceptual del problema, que se debe vincular con el tema
principal. Además, se establece un orden cronológico de los hechos, y se
establecen condiciones generales e informes previos para brindar más
información.

\subsection{Desarrollo}
Esta sección extiende el estudio del origen del problema, expone una propuesta
y análisis para la solución del problema,.

\subsection{Conclusión}
En esta sección, la causa del origen debe quedar claro. El tema debe quedar
cerrado el tema, su importancia y las posibles soluciones del mismo.

\subsection{Anexos}
Información extra como fotografías, planos, otros informes técnicos, etc.

\section{¿Por qué crees necesario seguir pasos para la elaboración del informe técnico?}
Los informes técnicos tratan de resolver un problema específico, generalmente
en situaciones donde la precisión es importante. Si los informes técnicos no
siguen una estructura detallada, serían muy heterogéneos entre sí. Además, esto
podría causar que, quien redacte el informe técnico, no siga una estructura
sólida.

Las consecuencias serían que el texto no sea comprendido correctamente, que no
se apunte al problema específico, y que, al no haber un órden, pierda el
sentido.

Esto se soluciona estandarizando la estructura del informe, facilitando la
producción del mismo. Lo asegura que se brinde la información necesaria.

\section{¿Qué es y para qué sirve la hoja de operaciones y la hoja de procesos?}
\subsection{Hoja de procesos}
La hoja de procesos es un documento en el que se describen ordenadamente todas
las operaciones, materiales y herramientas que se necesitan para fabricar un
objeto.

\subsection{Hoja de operaciones}
Es un documento en el que se explica de forma resumida como debe efectuarse
cada una de las operaciones, indicando materiales, herramientas, tiempo
necesario para llevarlas a cabo y persona responsable.

Se podría decir que la hoja de procesos describe de forma más general las
operacioens a realizar; la hoja de operaciones detalla cada proceso.

\end{document}