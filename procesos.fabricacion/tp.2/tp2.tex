\documentclass[a4paper, 12pt]{article}
\usepackage[spanish]{babel}

\title{Tiempos Modernos}
\author{Joaquín Gómez --- 5to ``A''}
\begin{document}
\maketitle
\section*{Actividad}
Detallar tres momentos de la película que sean significativos, y justificar su respuesta.

\section{Momento uno}
Como primer momento, al inicio del largometraje, se realiza una comparación: se muestra a
un rebaño de ovejas siendo trasladado, en una sola dirección y hacinadas; el siguiente fragmento
nos deja ver un rebaño de humanos, o sea, una multitud de trabajadores ingresando a la fábrica
en una forma similar.

En mi opinión, esto refleja una realidad de la que hoy somos testigos: estamos presos de un orden
mayor que nos rige, y somos rebaños de un pastor que, en algunos casos, no es más que la necesidad
de sobrevivir. Trabajamos para comer, vestirnos, tener un techo. Y en otros casos, somos ávidos y nos
sometemos para satisfacer carencias impuestas por el mismo sistema que nos domina.

En el contexto de la película, quienes trabajan lo hacen por necesidad. Podemos ver al padre de la protagonista (Paulette Goddard),
que es empleado de la fábrica donde también trabaja el protagonista (Charles Chaplin), luchando por mantener su trabajo.
Luego de su despido y el de los demás trabajadores, se realiza una huelga que dura poco antes de ser reprimida
por la policía, donde el padre de la chica a la que veíamos robar bananas en un muelle, fallece, dejando a la protagonista 
huérfana junto con sus hermanos.

\section{Momento dos}
Lo que me pareció más triste de la película es como el protagonista desea volver a la cárcel después de ser liberado,
es decir, vivía más cómodo preso que trabajando o estando en la calle.

Esto pasa en la realidad y hay varios relatos que cuentan como presos prefieren seguir siendo convictos, ya que no 
tienen salida laboral o tienen una vida más llevadera en prisión.

Sin embargo, nunca se va a comparar la vida de un preso con la libertad, siempre y cuando todo trabajo, por más duro o 
pesado que sea, tenga condiciones humanas para hacerlo.

En la antigüedad es comprensible que muchos trabajos hayan sido esclavizantes y peligrosos, mal pagos y con condiciones 
de vida precarias para sus empleados. Pero en la actualidad es inadmisible dadas las facilidades técnicas y tecnológicas 
que existen.

\section{Momento tres}
Como último momento y también al final del film, me pareció interesante como una persona que no encajaba en la sociedad 
laboral ``estándar'' pudo destacarse en un ámbito alternativo, y a pesar de que el final es abierto, podemos imaginar 
que los protagonistas pudieron subsistir interpretando el papel que la vida les dio naturalmente. En este caso, la actuación
o el baile.

\end{document}
