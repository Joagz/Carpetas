\documentclass{article}
\usepackage[spanish]{babel}

\title{Estado Gaseoso - Dilatación de gases}
\date{24 - 04 - 25}
\author{Joaquín Gómez, $5^{to}$ ``A''}

\begin{document}
\maketitle

\begin{enumerate}
    \item \textbf{Objeto}: comprobar el siguiente enunciado
          \begin{itemize}
              \item El volumen de los gases aumenta notablemente con la temperatura y viceversa.
              \item El recipiente, como todo sólido, resulta mucho menos sensible a los cambios de
                    temperatura.
              \item El aire no es un buen conductor de calor.
          \end{itemize}
    \item \textbf{Equipo}: según el esquema
    \item \textbf{Desarrollo}: según lo indicado
          Resultados:
          \begin{itemize}
            \item Temperatura inicial del aire: 24°C
            \item Temperatura final del aire: 27°C
            \item Temperatura del agua: 85°C
            \item El globo aumenta su volumen en forma muy visible
          \end{itemize}
\end{enumerate}

Al enfriarse el recipiente, luego de varios minutos la temperatura del aire
descendió y el globo se contrajo.

\section*{Conclusiones}
Se pudo comprobar que los gases son muy sensibles a los cambios de temperatura
que se reflejaron en los cambios de volumen. No así, el recipiente sólido.
También se comprobó que el aire es mal conductor del calor porque su
temperatura fue mucho más baja que la del vidrio.

\end{document}